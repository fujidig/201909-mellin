\documentclass{jsarticle}
\usepackage{amssymb}
\usepackage{amsmath}
\usepackage{amsthm}
\usepackage{framed}
\usepackage{braket}
\usepackage{bm}
\usepackage{mathrsfs}
\usepackage{mathabx}
\usepackage{accents}
\usepackage{tocloft}
\usepackage[dvipdfmx]{graphicx}
\usepackage{tikz}
\usepackage{url}
\usepackage{color}
\usepackage{xifthen}
\usepackage{xcolor}
\usepackage{framed}

\usetikzlibrary{positioning}
\usetikzlibrary{calc}
\usetikzlibrary{decorations.pathreplacing}

\newcommand{\scrN}{\mathcal{N}}
\newcommand{\N}{\mathbb{N}}
\newcommand{\Z}{\mathbb{Z}}
\newcommand{\Q}{\mathbb{Q}}
\newcommand{\R}{\mathbb{R}}
\newcommand{\C}{\mathbb{C}}
\newcommand{\Pfin}{\mathcal{P}_{\mathrm{fin}}}
\newcommand{\boldsig}{\undertilde{\boldsymbol{\Sigma}}}
\newcommand{\boldpi}{\undertilde{\boldsymbol{\Pi}}}
\newcommand{\bolddelta}{\undertilde{\boldsymbol{\Delta}}}
\newcommand{\range}{\operatorname{range}}
\newcommand{\dom}{\operatorname{dom}}
\newcommand{\Map}{\operatorname{Map}}
\newcommand{\append}{{}^\frown}
\newcommand{\Ein}{\operatorname{Ein}}

\renewcommand\Re{\operatorname{Re}}
\renewcommand\Im{\operatorname{Im}}

\newcommand{\needtocheck}[1][]{%
	\ifthenelse{\equal{#1}{}}{%
		\textcolor{blue}{[要チェック]}%
	}{%
		\textcolor{blue}{[要チェック: #1]}%
	}%
}

\newcommand{\todo}[1][]{%
	\ifthenelse{\equal{#1}{}}{%
		\textcolor{red}{[TODO]}%
	}{%
		\textcolor{red}{[TODO: #1]}%
	}%
}

\theoremstyle{definition}
\newtheorem{thm}{定理}
\newtheorem*{thm*}{定理}
\newtheorem{defi}[thm]{定義}
\newtheorem*{defi*}{定義}
\newtheorem{lem}[thm]{補題}
\newtheorem*{lem*}{補題}
\newtheorem{fact}[thm]{事実}
\newtheorem*{fact*}{事実}
\newtheorem*{formula*}{公式}
\newtheorem{prop}[thm]{命題}
\newtheorem*{claim*}{主張}
\newtheorem{exm}[thm]{例}
\newtheorem{rmk}[thm]{注意}
\newtheorem*{rmk*}{注意}
\newtheorem{cor}[thm]{系}
\newtheorem*{cor*}{系}
\renewcommand{\proofname}{証明}


\renewenvironment{leftbar}[1][\hsize]
{%
\def\FrameCommand
{%
{\color{black}\vrule width 1pt}%
\hspace{0pt}%must no space.
\fboxsep=\FrameSep\colorbox{white}%
}%
\MakeFramed{\hsize#1\advance\hsize-\width\FrameRestore}%
}
{\endMakeFramed}


\title{メリン変換を用いたある関数の漸近挙動の証明}
\author{でぃぐ (\texttt{@fujidig})}

\begin{document}
\maketitle

本稿では次の結果を証明する.

\begin{leftbar}
\begin{thm*}
\[ A(y) = \sum_{k \ge 1} \left(-1 + \exp \sum_{j \ge k} \frac{y^j}{j} \right) \]
とおく.
このとき
\[ A(e^{-x}) = \sum_{k \ge 1} \left(-1 + \exp \sum_{j \ge k} \frac{e^{-jx}}{j} \right) \]
は$x > 0$で$x \to 0$と近づけたとき
\[ A(e^{-x}) \sim e^{-\gamma} \frac{1}{x} \log \frac{1}{x} \]
となる.
\end{thm*}
\end{leftbar}

\subsection*{Step 0}

本稿で主に使う道具はメリン変換である.
関数$f(x)$のメリン変換は次で定義される:
\[ M[f(x)](s) = \int_0^\infty x^{s-1} f(x) dx. \]

\subsection*{Step 1}

まず,内側の和を積分で近似することにより
\begin{equation}
A(e^{-x}) \sim \sum_{k \ge 1} \left(-1 + \exp \int_{kx}^\infty \frac{e^{-u}}{u} du \right) \label{approx}
\end{equation}
を得る.

正確には
\[ A(e^{-x}) = \sum_{k \ge 1} \left(-1 + \exp \int_{kx}^\infty \frac{e^{-u}}{u} du \right) + O(1/x)\ \ \ \ (x \to 0) \]
となる.

式 (\ref{approx})の右辺を$\tilde{A}(x)$と書く.

ここで$E_1(x) = \int_{x}^\infty \frac{e^{-u}}{u} du$とおくと,
\[ \tilde{A}(x) = \sum_{k \ge 1} \left(-1 + \exp E_1(x) \right). \]

上の近似を正当化する.

\begin{leftbar}
\begin{formula*}[Euler--Maclaurinの公式]
関数$f$について
\[
\sum_{j=m}^n f(j) - \int_m^n f(t) dt = \frac{f(m)+f(n)}{2} + \int_m^n f'(t) (\{t\} - 1/2) dt.
\]
ここに$\{t\}$は実数$t$の小数部分.
\end{formula*}
\end{leftbar}

$f(t) = \frac{e^{-tx}}{tx}$に対してEuler--Maclaurinの公式を適用すると
\begin{align}
\frac{1}{x} \left(
\sum_{j \ge k} \frac{e^{-jx}}{j}
- \int_{kx}^\infty \frac{e^{-u}}{u} du
\right)
&= \sum_{j \ge k} \frac{e^{-jx}}{jx}
- \int_{k}^\infty \frac{e^{-ux}}{ux} du  \notag \\
& = \sum_{j \ge k} f(j)
- \int_{k}^\infty f(u) du  \notag \\
& = \frac{f(k)+f(\infty)}{2} + \int_k^\infty f'(t) (\{t\} - 1/2) dt \label{eulermac}
\end{align}

ここで$f(\infty)$は$0$であり,
\begin{align*}
\left| \int_k^\infty f'(t) (\{t\} - 1/2) dt \right|
&\le \frac{1}{2} \int_k^\infty |f'(t)| dt \\
&= - \frac{1}{2} \int_k^\infty f'(t) dt & \text{($f'$は常に負より)} \\
&= - \frac{1}{2} [f(t)]_{t=k}^\infty \\
&= \frac{1}{2} f(k)
\end{align*}
なので式 (\ref{eulermac})の絶対値は
\begin{align*}
\left|
\frac{f(k)+f(\infty)}{2} + \int_k^\infty f'(t) (\{t\} - 1/2) dt
\right| \le f(k) = \frac{e^{-k x}}{k x}.
\end{align*}
よって
\begin{align*}
\left|
\sum_{j \ge k} \frac{e^{-jx}}{j}
- \int_{kx}^\infty \frac{e^{-u}}{u} du
\right| \le \frac{e^{-k x}}{k}.
\end{align*}
この左辺の絶対値の中身を$\varepsilon(k, x)$とおく.

すると
\begin{align*}
\left|\exp \sum_{j \ge k} \frac{e^{-jx}}{j}
- \exp \int_{kx}^\infty \frac{e^{-u}}{u} du\right|
&= \left(\exp \int_{kx}^\infty \frac{e^{-u}}{u} du\right) |\exp \varepsilon(x, k) - 1| \\
&\le \exp E_1(k x) (e |\varepsilon(x, k)|) \\
&\le \exp E_1(k x) \frac{e \cdot e^{-k x}}{k}.
\end{align*}

今,
\begin{align*}
\sum_{k=1/x}^\infty \exp E_1(k x) \frac{e \cdot e^{-k x}}{k}
&\le \sum_{k=1/x}^\infty \exp (1) \frac{e \cdot e^{-k x}}{k} \\
&= e^2 \sum_{k=1/x}^\infty \frac{e^{-k x}}{k} \\
&= e^2 \int_{1/x-1}^\infty \frac{e^{-u x}}{u} du \\
&= e^2 \int_{1-x}^\infty \frac{e^{-t}}{t/x} \frac{dt}{x} \\
&= e^2 \int_{1-x}^\infty \frac{e^{-t}}{t} dt \\
&\le e^2 \int_{1/2}^\infty \frac{e^{-t}}{t} dt \\
&= e^2 E_1(1/2) \\
&= O(1).
\end{align*}

次に
\begin{align*}
\sum_{k=1}^{1/x} \exp E_1(k x) \frac{e \cdot e^{-k x}}{k}
&= \sum_{k=1}^{1/x} (\exp \int_{kx}^1 \frac{e^{-u}}{u} du) \exp E_1(1) \frac{e \cdot e^{-k x}}{k} \\
&= \sum_{k=1}^{1/x} \left(\exp \int_{kx}^1 \frac{e^{-u}}{u} du\right) \exp E_1(1) \frac{e \cdot e^{-k x}}{k} \\
&= e^{E_1(1)+1} \sum_{k=1}^{1/x} \left(\exp \int_{kx}^1 \frac{e^{-u}}{u} du\right) \frac{e^{-k x}}{k} \\
&\le e^{E_1(1)+1} \sum_{k=1}^{1/x} \left(\exp \int_{kx}^1 \frac{1}{u} du\right) \frac{1}{k} \\
&= e^{E_1(1)+1} \sum_{k=1}^{1/x} \frac{1}{kx} \frac{1}{k} \\
&\le e^{E_1(1)+1} \frac{1}{x} \sum_{k=1}^{\infty} \frac{1}{k^2} \\
&\le e^{E_1(1)+1} \zeta(2) \frac{1}{x} \\
&= O(1/x)
\end{align*}

したがって
\begin{align*}
|A(e^{-x}) - \tilde{A}(x)|
&\le \sum_{k \ge 1} \left|\exp \sum_{j \ge k} \frac{e^{-jx}}{j} - \exp \int_{kx}^\infty \frac{e^{-u}}{u} du \right| \\
&= O(1/x)\ \ \ \ (x \to 0)
\end{align*}
がわかる.



以降$\tilde{A}(x)$の漸近挙動を調べる.

次の公式は今後しばしば使う.

\begin{leftbar}
\begin{formula*}
\begin{equation}
E_1(x) = -\gamma-\log x + \Ein(x) \label{e1_ein_fml}
\end{equation}
ただし,$\Ein(x) = \int_0^x \frac{1 - e^{-u}}{u} du$.
\end{formula*}
\end{leftbar}

\subsection*{Step 2}

今,$\Re(s)>1$としておく.

次のメリン変換の公式を使う:

\begin{leftbar}
\begin{formula*}
\[ M\left[\sum_{k \ge 1} f(k x)\right](s) = \zeta(s)M[f](s). \]
\end{formula*}
\end{leftbar}

すると
\[ M[\tilde{A}](s) = \zeta(s) M\left[ -1 + \exp E_1(x) \right](s) \]
となる.

この公式では積分と総和の交換をしている.それを正当化する.まず公式の証明を今の場合に限定して見る.

\begin{align*}
M\left[\sum_{k \ge 1} \left(-1 + \exp E_1(k x) \right)\right]
&= \int_0^\infty x^{s-1} \left(\sum_{k \ge 1} \left(-1 + \exp E_1(k x) \right)\right) dx \\
&= \sum_{k \ge 1} \int_0^\infty x^{s-1} \left(-1 + \exp E_1(k x) dx \right)  & (\ast) \\
&=  \sum_{k \ge 1} \int_0^\infty \left(\frac{y}{k}\right)^{s-1} \left(-1 + \exp E_1(y) \right) \frac{dy}{k} \\
&=  \left(\sum_{k \ge 1} k^{-s}\right) \left(\int_0^\infty y^{s-1} \left(-1 + \exp E_1(y) \right) dy\right) \\
&= \zeta(s) \int_0^\infty y^{s-1} \left(-1 + \exp E_1(y) \right) dy
\end{align*}

$(\ast)$が積分と総和の交換をした部分である.この部分を正当化するには,
\begin{align*}
\sum_{k \ge 1} \int_0^\infty \left|x^{s-1} \left(-1 + \exp E_1(k x) \right)\right| dx
\end{align*}
が有限になればよい.

まず上と同じ変形をすることにより
\begin{align*}
\sum_{k \ge 1} \int_0^\infty \left|x^{s-1} \left(-1 + \exp E_1(k x) \right)\right| dx
&= \zeta(s) \int_0^\infty |y^{s-1}| \left(-1 + \exp E_1(y) \right) dy \\
&= \zeta(s) \sum_{n \ge 1} \frac{1}{n!} \int_0^\infty y^{\Re(s)-1} \left(E_1(y)\right)^n dy. \\
\end{align*}

ここで$0 < y < 1$なら
\begin{align*}
E_1(y) = -\gamma-\log x+\Ein(y) \le 1-\gamma-\log y = -\log \frac{y}{e^{1-\gamma}}
\end{align*}
である.

$c = 1/e^{1-\gamma}$とおく.すると$0$から$1$までの積分は
\begin{align*}
\sum_{n \ge 1} \frac{1}{n!} \int_0^1 y^{\Re(s)-1} (E_1(y))^n dy
&= \sum_{n \ge 1} \frac{1}{n!} \int_0^1 y^{\Re(s)-1} (- \log cy)^n dy \\
&= \sum_{n \ge 1} \frac{1}{n!} \int_0^\infty  \left(\frac{e^{-z}}{c}\right)^{\Re(s)-1} z^n \frac{e^{-z}dz}{c} \\
&= \frac{1}{c^{\Re(s)}}\sum_{n \ge 1} \frac{1}{n!} \int_0^\infty  e^{-\Re(s) z} z^n dz \\
&= \frac{1}{c^{\Re(s)}}\sum_{n \ge 1} \frac{1}{n!} \int_0^\infty  e^{-w} \left(\frac{w}{\Re(s)}\right)^n \frac{dw}{\Re(s)} \\
&= \frac{1}{c^{\Re(s)}}\sum_{n \ge 1} \frac{1}{n!} \frac{1}{\Re(s)^{n+1}} \int_0^\infty  e^{-w} w^n dw \\
&= \frac{1}{c^{\Re(s)}}\sum_{n \ge 1} \frac{1}{n!} \frac{1}{\Re(s)^{n+1}} \Gamma(n+1) \\
&= \frac{1}{c^{\Re(s)}}\sum_{n \ge 1} \frac{1}{\Re(s)^{n+1}}  \\
&= \frac{1}{c^{\Re(s)}} \frac{1}{\Re(s)^2 (1-1/\Re(s))}  \\
\end{align*}

次に$y > 1$なら
\begin{align*}
E_1(y) = \int_{y}^\infty \frac{e^{-u}}{u} du \le \int_{y}^\infty e^{-u} du = e^{-y}
\end{align*}
だから,$1$から$\infty$までの積分は,
\begin{align*}
\sum_{n \ge 1} \frac{1}{n!} \int_1^\infty y^{\Re(s)-1} (E_1(y))^n dy
&\le \sum_{n \ge 1} \frac{1}{n!} \int_1^\infty y^{\Re(s)-1} e^{-ny} dy \\
&= \sum_{n \ge 1} \frac{1}{n!} \int_n^\infty \left(\frac{z}{n}\right)^{\Re(s)-1} e^{-z} \frac{dz}{n} \\
&= \sum_{n \ge 1} \frac{1}{n!} \frac{1}{n^{\Re(s)}}\int_n^\infty z^{\Re(s)-1} e^{-z} dz \\
&= \sum_{n \ge 1} \frac{1}{n!} \frac{1}{n^{\Re(s)}}\int_n^\infty z^{\Re(s)-1} e^{-z} dz \\
&\le \sum_{n \ge 1} \frac{1}{n!} \frac{1}{n^{\Re(s)}} \Gamma(\Re(s)) \\
&\le \sum_{n \ge 1} \frac{1}{n!} \Gamma(\Re(s)) \\
&= e \Gamma(\Re(s))
\end{align*}

よって,
\begin{align*}
\sum_{k \ge 1} \int_0^\infty \left|x^{s-1} \left(-1 + \exp E_1(k x) \right)\right| dx
\le \zeta(s) \left(\frac{1}{c^{\Re(s)}} \frac{1}{\Re(s)^2 (1-1/\Re(s))} + e \Gamma(\Re(s))\right).
\end{align*}
ゆえに積分と総和の交換は正当化された.

\subsection*{Step 3}
メリン変換の定義に戻り,部分積分を行う.引き続き,$\Re(s)>1$としておく.すると
\begin{align}
M\left[ -1 + \exp E_1(x) \right](s) &= \int_0^\infty x^{s-1} \left(-1 + \exp \int_{x}^\infty \frac{e^{-u}}{u} du\right) dx \notag \\
&= \int_0^\infty \left(\frac{x^s}{s}\right)' \left(-1 + \exp \int_{x}^\infty \frac{e^{-u}}{u} du\right) dx \notag \\
&= - \int_0^\infty \frac{x^s}{s} \left( \exp \int_{x}^\infty \frac{e^{-u}}{u} du\right) \left(-\frac{e^{-x}}{x}\right) dx \notag \\
&= \frac{1}{s} \int_0^\infty x^{s-1} e^{-x} \left( \exp \int_{x}^\infty \frac{e^{-u}}{u} du\right) dx \notag \\
&= \frac{1}{s} \int_0^\infty x^{s-1} e^{-x} \exp E_1(x) dx \label{step3_conclusion}
\end{align}
を得る.ここで部分積分の境界項は消えることに注意する.

実際,
\begin{align*}
\lim_{x\to\infty} x^s (-1+\exp \int_x^\infty \frac{e^{-u}}{u}du)
&= \lim_{x\to\infty} \frac{-1+\exp \int_x^\infty \frac{e^{-u}}{u}du}{x^{-s}} \\
&= \lim_{x\to\infty} \frac{(\exp \int_x^\infty \frac{e^{-u}}{u}du) (-\frac{e^{-x}}{x})}{-s x^{-s-1}}  & \text{(ロピタル)} \\
&= \lim_{x\to\infty} \frac{1}{s} e^{-x} x^{s} \exp \int_x^\infty \frac{e^{-u}}{u}du \\
&= \frac{e^{-\gamma}}{s} \lim_{x\to\infty} e^{-x} x^{s-1} \exp \Ein(x)
\end{align*}
であり,
\begin{align*}
|e^{-x} x^{s-1} \exp \Ein(x)| &\le e^{-x} x^{\Re(s)-1} e x \\
&= e e^{-x} x^{\Re(s)} \\
&\to 0 \ \ \ \ (x \to \infty)
\end{align*}
なので$x=\infty$の境界項は消える.

\begin{align*}
\lim_{x\to +0} x^s (-1+\exp \int_x^\infty \frac{e^{-u}}{u}du)
&= \lim_{x\to +0} x^s \exp \int_x^\infty \frac{e^{-u}}{u}du \\
&= e^{-\gamma} \lim_{x\to +0} x^{s-1} \exp \Ein(x) \\
&= e^{-\gamma} (\lim_{x\to +0} x^{s-1}) (\lim_{x\to +0} \exp \Ein(x)) \\
&= e^{-\gamma} \cdot 0 \cdot 1 \\
&= 0
\end{align*}
なので$x=0$の境界項も消える.


\subsection*{Step 4}

式 (\ref{e1_ein_fml})をStep 3の式 (\ref{step3_conclusion})に代入し,
\begin{align*}
M\left[ -1 + \exp \int_{x}^\infty \frac{e^{-u}}{u} du \right](s) &= \frac{1}{s} \int_0^\infty x^{s-1} e^{-x}  \exp \left(-\gamma-\log x+\Ein(x)\right)  dx \\
&= \frac{e^{-\gamma}}{s} \int_0^\infty x^{s-2} e^{-x}  \exp \Ein(x) dx \\
\end{align*}
を得る.

\subsection*{Step 5}
Step 4で得た結果をStep 2の式に代入する.すると,
\begin{align*}
M[\tilde{A}](s) &= \frac{e^{-\gamma}}{s} \zeta(s) \int_0^\infty x^{s-2} e^{-x}  \exp \Ein(x) dx \\
&= \frac{e^{-\gamma}}{s} \zeta(s) g(s-1)
\end{align*}
を得る.ここに
\[ g(s) = \int_0^\infty x^{s-1} e^{\Ein(x)-x} dx. \]

\subsection*{Step 6}
$\Re(s) > 0$とする.
$g(s) = \int_0^\infty x^{s-1} f(x) dx$の形の関数の微分を考える.形式的に積分記号化の微分を行うと,
\begin{equation}
\frac{d}{ds} g(s) = \int_0^\infty x^{s-1} f(x) \log x dx \label{dgds}
\end{equation}
となる.どういう条件の下でこの変形が正当化できるか考える.この式の右辺を$\tilde{g}(s)$と書くと
\begin{align*}
\left|\frac{g(s+\Delta s) - g(s)}{h} - \tilde{g}(s)\right| &\le \frac{1}{|\Delta s|} \int_0^\infty |x^{s-1}| |f(x)| |x^{\Delta s}-1-\Delta s \log x| dx \\
 &\le \frac{1}{|\Delta s|} \int_{-\infty}^\infty |e^{sy}| |f(e^y)| |e^{\Delta sy}-1-\Delta sy| dy \\
 &\le \frac{1}{|\Delta s|} \int_{-\infty}^\infty |e^{sy}| |f(e^y)| |\sum_{n\ge 2} (\Delta sy)^n / n!| dy \\
 &= \frac{1}{|\Delta s|} \sum_{n\ge 2} \frac{|\Delta s|^n}{n!} \int_{-\infty}^\infty |e^{sy}| |f(e^y)| |y|^n dy\\
 &= \frac{1}{|\Delta s|} \sum_{n\ge 2} \frac{|\Delta s|^n}{n!} \int_{0}^\infty |x^{s-1}| |f(x)| |\log x|^n dx \\
 &= |\Delta s| \sum_{n\ge 2} \frac{|\Delta s|^{n-2}}{n!} \int_{0}^\infty |x^{s-1}| |f(x)| |\log x|^n dx & \text{($\heartsuit$)} \\
\end{align*}

よって,式(\ref{dgds})が成り立つためには($\heartsuit$)が$\Delta s \to 0$で$0$に収束すればよい.

\subsection*{Step 7}
$g(s)$が$\Re(s) > -1$に有理型接続されることを示す.
まず,$\Re(s) > 0$とすると,部分積分により
\begin{align*}
g(s) &= \int_0^\infty x^{s-1} e^{\Ein(x)-x} dx \\
&= -\frac{1}{s} \int_0^\infty x^s e^{\Ein(x)-x} \left(\frac{1-e^{-x}}{x} - 1\right) dx \\
&= \frac{1}{s (s+1)} \int_0^\infty x^{s+1} e^{\Ein(x)-x} \varphi(x) dx \\
&= \frac{1}{s (s+1)} \int_0^\infty x^{s} e^{\Ein(x)-x} \psi(x) dx \\
\end{align*}
ここに$x \varphi(x) = \psi(x) = x - 2 + 3 e^{-x} + \frac{e^{-2x} - e^{-x}}{x}$.

$\Re(s) > -1$で$\int_0^\infty x^{s} e^{\Ein(x)-x} \psi(x) dx$が正則であればよい.つまり,
$\Re(s) > 0$で
\[
h(s) = \int_0^\infty x^{s-1} e^{\Ein(x)-x} \psi(x) dx
\]
が正則であればよい.そうすれば$g(s)$は$\Re(s) > -1$で
\[
\frac{1}{s(s+1)} h(s+1)
\]
と有理型接続される.

Step 6で行った議論より
\begin{align}
|\Delta s| \sum_{n\ge 2} \frac{|\Delta s|^{n-2}}{n!} \int_{0}^\infty |x^{s-1}| |e^{\Ein(x)-x}| |\psi(x)| |\log x|^n dx \label{dgdserr}
\end{align}
が$\Delta s \to 0$で$0$に収束することを示せばよい.

中の積分の積分範囲を$0$から$1$にしたものを考える.$|e^{\Ein(x)-x}| |\psi(x)|$はこの範囲で有界だから,上界を$c$とすると
\begin{align*}
\int_0^1 |x^{s-1}| |e^{\Ein(x)-x}| |\psi(x)| |\log x|^n dx
&\le c \int_0^1 x^{\Re(s)-1} (-\log x)^n dx \\
&= c \int_0^\infty e^{\Re(s) y} y^n dy\ \ \ \ (y = - \log x) \\
&= \frac{c}{\Re(s)^{n+1}} \int_0^\infty e^{z} z^n dz \\
&= \frac{c n!}{\Re(s)^{n+1}}.
\end{align*}
他方で,積分範囲を$1$から$\infty$にしたものは,
\begin{align}
\int_1^\infty |x^{s-1}| |e^{\Ein(x)-x}| |\psi(x)| |\log x|^n dx
&\le \int_1^\infty |x^{s-1}| |e^{\Ein(x)-x}| |\psi(x)| ((n/e) x^{1/n})^n dx \notag \\
&\le \left(\frac{n}{e}\right)^n \int_1^\infty x^{\Re(s)} |e^{\Ein(x)-x}| |\psi(x)| dx \notag \\
&\le \left(\frac{n}{e}\right)^n (\int_1^\infty x^{\Re(s)+1} |e^{\Ein(x)-x}| dx + 7\int_1^\infty x^{\Re(s)} |e^{\Ein(x)-x}| dx) \label{int1inf}
\end{align}
最初の式変形は$\log x \le (n/e) x^{1/n}$を使った.最後の式変形は$|\psi(x)| \le x + 7$を使った.

ここで
\begin{align*}
\int_1^\infty x^t e^{\Ein(x)-x} dx
&\le \int_1^\infty x^t e^{1+\log x - x} dx \\
&\le e \int_1^\infty x^{t+1} e^{- x} dx \\
&\le e \Gamma(t+2)
\end{align*}
なことを使えば,

\begin{align*}
\text{式 (\ref{int1inf})}
&\le \left(\frac{n}{e}\right)^n e (\Gamma(\Re(s)+3) + 7 \Gamma(\Re(s)+2)) \\
&\le e n!  (\Gamma(\Re(s)+3) + 7 \Gamma(\Re(s)+2))
\end{align*}

したがって,式 (\ref{dgdserr})の評価は
\begin{align*}
&|\Delta s| \sum_{n\ge 2} \frac{|\Delta s|^{n-2}}{n!} \int_{0}^\infty |x^{s-1}| |e^{\Ein(x)-x}| |\psi(x)| |\log x|^n dx \\
&\ \ \ \ \le |\Delta s| \sum_{n\ge 2} \frac{|\Delta s|^{n-2}}{n!} n! (\frac{c}{\Re(s)^{n+1}} + e \Gamma(\Re(s)+3) + 7 e \Gamma(\Re(s)+2)) \\
&\ \ \ \ = |\Delta s| \sum_{n\ge 2} |\Delta s|^{n-2} (\frac{c}{\Re(s)^{n+1}} + e \Gamma(\Re(s)+3) + 7 e \Gamma(\Re(s)+2)) \\
&\ \ \ \ = |\Delta s| \left(\frac{c}{\Re(s)^3} \frac{1}{1-|\Delta s|/\Re(s)} + (e \Gamma(\Re(s)+3) + 7 e \Gamma(\Re(s)+2)) \frac{1}{1-|\Delta s|} \right) \\
&\ \ \ \ \to 0\ \ \ \ (\Delta s \to 0)
\end{align*}

よって$g(s)$は$\Re(s) > -1$に有理型接続される.

\subsection*{Step 8}
以上より
\begin{align*}
M[\tilde{A}](s) &= \frac{e^{-\gamma}}{s} \zeta(s) g(s-1) \\
&= \frac{e^{-\gamma}}{s} \zeta(s) \frac{1}{(s-1) s} h(s) \\
&= \frac{e^{-\gamma}}{s^2 (s-1)} \zeta(s) h(s)
\end{align*}
を得る.ここに$h(s)$は$\Re(s) > 0$で正則な関数.

$1/s^2, h(s)$のテイラー展開と$\zeta(s)$のローラン展開は
\begin{align*}
1/s^2 &= 1 - 2(s - 1) + O((s-1)^2) \ \ \ \ (s \to 1) \\
h(s) &= h(1) + h'(1)(s - 1) + O((s-1)^2) \ \ \ \ (s \to 1) \\
\zeta(s) &= \frac{1}{s-1} + \gamma + O(s-1) \ \ \ \ (s \to 1)
\end{align*}
なことを使うと,$\Re(s) > 0$で$M[\tilde{A}](s)$は
\begin{align}
M[\tilde{A}](s) \asymp  \frac{e^{-\gamma} h(1)}{(s-1)^2} + \frac{e^{-\gamma} h(1) (-2+\gamma) + h'(1)}{s-1} \label{mas}
\end{align}
と有理型接続される.ここに$\asymp$は差が正則関数の意味.

$h(1) = 1$を示す.
\[
h(1) = \int_0^\infty e^{\Ein(x)-x} \psi(x) dx
\]
であるが,$\psi(x)$の定義をよく見ると
\[
e^{\Ein(x)-x} \psi(x) = x \frac{d^2}{dx^2} e^{\Ein(x)-x}
\]
であるので,
\begin{align*}
h(1) &= \int_0^\infty x \frac{d^2}{dx^2} e^{\Ein(x)-x} dx \\
&= \left[x \frac{d}{dx} e^{\Ein(x)-x}\right]_{x=0}^\infty - \int_0^\infty \frac{d}{dx} e^{\Ein(x)-x} dx \\
&= \left[x \frac{d}{dx} e^{\Ein(x)-x} - e^{\Ein(x)-x}\right]_{x=0}^\infty \\
&= \left[x e^{\Ein(x)-x} \left(\frac{1-e^{-x}}{x}-1\right) - e^{\Ein(x)-x}\right]_{x=0}^\infty \\
&= 1
\end{align*}
となる.式 (\ref{mas})に$h(1)=1$を代入すると
\begin{align*}
M[\tilde{A}](s) \asymp  \frac{e^{-\gamma}}{(s-1)^2} + \frac{e^{-\gamma} (-2+\gamma) + h'(1)}{s-1} \label{mas}.
\end{align*}


\subsection*{Step 9}
メリン変換の漸近挙動の対応の定理を使う.

\begin{defi*}
実数$a < b$について$\langle a, b \rangle = \{s \in \C \mid a < \Re(s) < b \}$とおく.
メリン変換$Mf(s)$の値が$\langle a, b \rangle$内で存在するような最大の$\langle a, b \rangle$をこのメリン変換のfundamental stripという.
\end{defi*}

\begin{leftbar}
\begin{thm*}
$f(x)$を$(0, \infty)$で連続な関数とし,そのメリン変換$Mf(s)$はfundamental strip $\langle \alpha, \beta \rangle$を持つとする.
ある$\alpha' < \alpha$について,$Mf(s)$が$\langle \alpha', \beta \rangle$に有理型接続できると仮定し,そこで極は有限個で,$\Re(s) = \gamma$で解析的とする.
また,$\eta \in (\alpha, \beta)$と$r > 1$が存在して,$\alpha' \leq \Re(s) \leq \eta$で$|s| \to \infty$としたとき
\[ Mf(s) = O(|s|^{-r}) \]
となることを仮定する.
このときもし,$s \in \langle \alpha', \alpha \rangle$について
\[ Mf(s) \asymp \sum_{(\xi,k)\in A} d_{\xi,k} \frac{1}{(s-\xi)^k} \]
ならば
\[ f(x) = \sum_{(\xi,k)\in A} d_{\xi,k} \left( \frac{(-1)^{k-1}}{(k-1)!} x^{-\xi} (\log x)^{k-1} \right) + O(x^{-\alpha'}) \]
となる.
\end{thm*}
\end{leftbar}

定理を$f = \tilde{A}, \alpha=1, \beta=\infty, \alpha'=0.501, \eta=2$で適用すると
\[
\tilde{A}(x) = - e^{-\gamma} x^{-1} \log x + (e^{-\gamma} (-2+\gamma) + h'(1)) x^{-1} + O(x^{-0.501})
\]
を得る.

ここで定理の$Mf(s)$が$|s| \to \infty$で減衰するという仮定を確かめよう.$h$の定義より
\[
|h(s)| \le |h(\Re(s))|
\]
であり,$|h(s)|$はその連続性より$\alpha' \le s \le \eta$で有界なので$|h(s)|$は$\alpha' \le \Re(s) \le \eta$で有界.
また,$\zeta(s)$については$\Re(s)>1/2$で$\zeta(s) = O(\Im(s)^{1/2})$であることが知られている.これについてはPaul T BetemanとHarold G Diamondの``Analytic Number Theory: An Introductory Course''を見よ.

よって定理の$Mf(s)$が減衰することの仮定は確かめられた.

したがって,
\[
\tilde{A}(x) \sim e^{-\gamma} \frac{1}{x} \log \frac{1}{x}\ \ \ \ (x \to 0, x > 0)
\]
を得る.
よって,
\[
A(e^{-x}) \sim e^{-\gamma} \frac{1}{x} \log \frac{1}{x}\ \ \ \ (x \to 0, x > 0).
\]

\todo[$x$を複素数で$0$に近づけるときの話]

\end{document}